% !TEX root = Ausarbeitung.tex
\section{Anhang}
\label{sec:anhang}

\subsection{Mögliche Einsatzmöglichkeiten von Containern an der Hochschule Albstadt-Sigmaringen}
\label{app:EinsatzHSAlbSig}

Im folgenden soll betrachtet werden, welche Einsatzmöglichkeiten sich für Container in der IT-Infrastruktur der Hochschule Albstadt-Sigmaringen anbieten und welche Vorteile dies mit sich bringen würde. 

Prädestiniert für den Einsatz sind die Server-Dienste der Hochschule.
Hier wäre denkbar den in \Abschnitt{Cluster} vorgestellten Ansatz eines Serververbunds in Verbindung mit einem Cluster Manager wie Docker Swarm oder Kubernetes zu verwenden.
Die einzelnen Applikationen wie z.B. das E-Learning System Ilias, die CentOS Instanzen, die Bibliotheksdienste und die Website könnten dann in mehreren Container-Instanzen laufen.
Der Cluster-Manager würde dynamisch die Last auf die verschiedenen Maschinen verteilen und könnte Instanzen mit Programmfehlern erkennen sowie schnell neustarten.

Außerdem wäre es möglich, Lastspitzen abzufangen, indem Serverkapazitäten anderer Bildungseinrichtungen genutzt werden, um dort bei Bedarf Containerinstanzen zu starten.

\subsection{Begründung der ausgewählten Literatur}
\label{app:BegruendungLiteratur}
% Container sind sehr neu
Zur Verfügung stand lediglich sehr aktuelle Literatur, da die Containervirtualisierung erst seit Erscheinen von Docker im Jahr 2013 in der IT-Branche an Bedeutung gewonnen hat. Daher sind auch viele der hier betrachteten Werkzeuge erst in den vergangen Jahren entwickelt worden.

% Kurzer Zeitraum der Bearbeitung
In  Anbetracht des kurzen Zeitraumes, der den Autoren zur Verfügung stand, konnte keine Fernleihe durchgeführt werden.
Eine Vorbestellung der Literatur war daher ebenfalls nicht möglich.
Am ersten Tag der Bearbeitung des Artikels stand außerdem die Bibliothek aufgrund des Betriebsausflugs nicht zur Verfügung, weshalb auch nicht auf die physischen Medien zurückgegriffen werden konnte.
Deshalb hat sich die Bücher- bzw. Artikelauswahl auf die über die Hochschule verfügbaren digitalen Medien beschränkt.

% Dokumentationen
Die  Literaturauswahl umfasst außerdem Dokumentationen der gängisten Software zum Thema Container-Technologie.
Diese wurde zum Verständnis des Aufbaus und der Nutzung des jeweiligen Werkzeugs genutzt. 
Die Dokumentationen sind online bzw. zusammen mit dem jeweiligen Source Code verfügbar und werden von den Entwicklern zur  Verfügung gestellt. 
Daher handelt sich bei den Dokumentationen um eine verlässliche Quelle über das jeweilige Werkzeug.


% Offizelle Homepage
Auf den offizellen Webseiten der verschiedenen Hersteller und Projekten werden von den Entwicklern oder Firmen offizelle Informationen publiziert oder auch oben genannte Dokumentationen veröffentlicht. Der Inhalt der Webseiten kann als verlässliche Quelle angesehen werden, da hier der Ersteller des Produkts direkt veröffentlicht.


% Blogs
Blog Einträge dienten den Autoren als Ideengeber für einen Teil des Inhalts der vorliegenden Arbeit.
Da diese am Puls der Zeit sind, zeigen sie aktuelle Trends und populäre Software zum Thema Container-Technologie auf.
Ein Blog wird nicht überprüft und stellt daher selbstverständlich keine zuverlässige Quelle dar.
Zur weiteren Recherche wurden aufgrund dessen wissenschaftlich verlässliche Quellen verwendet.