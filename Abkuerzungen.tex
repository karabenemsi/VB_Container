% !TEX root = Projektdokumentation.tex

% Es werden nur die Abkürzungen aufgelistet, die mit \ac definiert und auch benutzt wurden. 
%
% \acro{VERSIS}{Versicherungsinformationssystem\acroextra{ (Bestandsführungssystem)}}
% Ergibt in der Liste: VERSIS Versicherungsinformationssystem (Bestandsführungssystem)
% Im Text aber: \ac{VERSIS} -> Versicherungsinformationssystem (VERSIS)

% Hinweis: allgemein bekannte Abkürzungen wie z.B. bzw. u.a. müssen nicht ins Abkürzungsverzeichnis aufgenommen werden
% Hinweis: allgemein bekannte IT-Begriffe wie Datenbank oder Programmiersprache müssen nicht erläutert werden,
%          aber ggfs. Fachbegriffe aus der Domäne des Prüflings (z.B. Versicherung)

% Die Option (in den eckigen Klammern) enthält das längste Label oder
% einen Platzhalter der die Breite der linken Spalte bestimmt.
\begin{acronym}[WWWWW]
	\acro{API}{Application Programming Interface}
	\acro{CI}{Continous Integration}
	\acro{IDE}{Integrated Development Environment}
	\acro{SDK}{Software Development Kit}
	\acro{SaSS}{Software as a Service}
    \acro{UI}{Benutzeroberfläche (engl.: "`User Interface"')}
    \acro{VCS}{Version Control System}
    \acro{CI}{Kontinuierliche Integration (engl.: "Continuous Integration")}
    \acro{VM}{Virtuelle Maschine}
    \acro{CPU}{Central Processing Unit}
    \acro{SPARC}{Scalable Processor ARChitecture}
    \acro{LXC}{Linux Containers}
\end{acronym}
