% !TEX root = Projektdokumentation.tex

% Es werden nur die Abkürzungen aufgelistet, die mit \ac definiert und auch benutzt wurden. 
%
% \acro{VERSIS}{Versicherungsinformationssystem\acroextra{ (Bestandsführungssystem)}}
% Ergibt in der Liste: VERSIS Versicherungsinformationssystem (Bestandsführungssystem)
% Im Text aber: \ac{VERSIS} -> Versicherungsinformationssystem (VERSIS)

% Hinweis: allgemein bekannte Abkürzungen wie z.B. bzw. u.a. müssen nicht ins Abkürzungsverzeichnis aufgenommen werden
% Hinweis: allgemein bekannte IT-Begriffe wie Datenbank oder Programmiersprache müssen nicht erläutert werden,
%          aber ggfs. Fachbegriffe aus der Domäne des Prüflings (z.B. Versicherung)

% Die Option (in den eckigen Klammern) enthält das längste Label oder
% einen Platzhalter der die Breite der linken Spalte bestimmt.
\begin{acronym}[WWWWW]
	\acro{API}{Application Programming Interface}
	\acro{CSS}{Cascading Style Sheets}
	\acro{CSV}{Comma Separated Value}
	\acro{CI}{Continous Integration}
	\acro{EPK}{Ereignisgesteuerte Prozesskette}
	\acro{ERM}{En\-ti\-ty-Re\-la\-tion\-ship-Mo\-dell}
	\acro{HTML}{Hypertext Markup Language}\acused{HTML}
	\acro{LDAP}{Lightweight Directory Access Protocol}
    \acro{SASS}{Syntactically Awesome Stylesheets}
	\acro{IDE}{Integrated Development Environment}
	\acro{SDK}{Software Development Kit}
	\acro{SaSS}{Software as a Service}
	\acro{SQL}{Structured Query Language}
	\acro{SVN}{Subversion}
	\acro{UML}{Unified Modeling Language}
	\acro{XML}{Extensible Markup Language}
	\acro{FVSS}{Ferdinand-von-Steinbeis-Schule}
    \acro{UI}{Benutzeroberfläche (engl.: "`User Interface"')}
    \acro{SCSS}{Sassy CSS}
    \acro{VCS}{Version Control System}
    \acro{MVC}{Model-View-Controller}
    \acro{CI}{Kontinuierliche Integration (engl.: "Continuous Integration")}
    \acro{DI}{Dependency Injection}
    \acro{DRY}{Wiederhole dich nicht (engl.: Don't repeat yourself)}
\end{acronym}
