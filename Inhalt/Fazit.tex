% !TEX root = ../Ausarbeitung.tex
\section{Fazit und Ausblick} 
\label{sec:Fazit}
In den letzten Jahren stieg die Nutzung von Containern rasant an und viele Unternehmen investieren massiv in diese Technologie (vgl. \Abschnitt{AktuelleLage}). Der ausschlaggebende Grund hierfür sind der verringerte Overhead, die Performance-Vorteile, und das vereinfachte Deployment von Anwendungen.  Im Bereich der Softwareentwicklung, bringen Container einen großen Vorteil. Es ist möglich innerhalb von Sekunden ein System mit den nötigen Eigenschaften aufzusetzen, um die Anwendung darauf auszuführen. Außerdem kann so sichergestellt werden, dass Test- und Produktivsystem absolut identisch sind und keine unerwarteten Effekte auftreten. Durch die immer simpleren Containertechnologien können Unternehmen mit wenig Budget und Personal die Containerisierung ihres Unternehmens vorantreiben und die Vorteile dieser ausschöpfen. Somit ist auch in den nächsten Jahren mit dem vermehrten Einsatz von Containertechnologien im Business Bereich zu rechnen. 

Container sind kein Ersatz für virtuelle Maschinen, sie können sogar sinnvoll kombiniert werden. Dabei ist es vor allem zu empfehlen den Docker-Host zu virtualisieren, sodass dieser bei einem Hardwareausfall trotzdem weiter laufen kann und bei Wartungen der Dienst weiter zur Verfügung gestellt werden kann.

