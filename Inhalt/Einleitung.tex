% !TEX root = ../Projektdokumentation.tex
\section{Einleitung}
\label{sec:Einleitung}

Bis kurz vor der Jahrtausendwende führte die Virtualisierung von Servern ein Schattendasein und jeder Service wurde auf einem dedizierten Server zur Verfügung gestellt. Dabei war es keine Seltenheit, dass Server sehr gering ausgelastet waren und der Ausfall eines nicht redundanten Servers einen Totalausfall bedeutete. Um diese und weitere Probleme zu lösen, gewann die Virtualisierung von Servern zum Anfang des neuen Jahrtausends immer mehr an Bedeutung und ist heutzutage ein fester Bestandteil vieler großer Unternehmen. Doch auch die Virtualisierung von Servern birgt noch verbesserbare Nachteile. So entsteht durch das Betriebssystem der virtuellen Maschinen ein deutlicher Overhead, da diese zur Laufzeit etliche Services benötigen.  