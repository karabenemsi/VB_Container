% !TEX root = ../Ausarbeitung.tex
\section{Einsatzszenarien von Containern an der Hochschule}
\label{sec:HS}
Im Folgenden soll betrachtet werden, welche Einsatzmöglichkeiten sich für Container in der IT-Infrastruktur der Hochschule Albstadt-Sigmaringen anbieten und welche Vorteile dies mit sich bringen würde. 

Da Container keine Schnittstelle für grafische Oberflächen bieten, sind die Server-Dienste der Hochschule für den Einsatz prädestiniert.
Hier wäre denkbar den in \Abschnitt{Cluster} vorgestellten Ansatz eines Serververbunds in Verbindung mit einem Cluster Manager wie Docker Swarm oder Kubernetes zu verwenden.
Die einzelnen Applikationen wie z.B. das E-Learning System Ilias, die CentOS Instanzen, die Bibliotheksdienste und die Website könnten dann in mehreren Container-Instanzen laufen.
Der Cluster-Manager würde dynamisch die Last auf die verschiedenen Maschinen verteilen und könnte Instanzen mit Programmfehlern erkennen sowie schnell neustarten.

Außerdem wäre es möglich Lastspitzen abzufangen, indem Serverkapazitäten anderer Bildungseinrichtungen genutzt werden, um dort bei Bedarf Containerinstanzen zu starten.

\textbf{Beispiel:}\newline
Ein konkreter Anwendungsfall für die Hochschue Albstadt-Sigmaringen ist die Containerisierung der Oracle Datenbank. Die Datenbank wird im Rahmen mehreren Vorlesungen und Praktika von Professoren und Studenten genutzt.
Dabei ist die Zahl der Zugriffe auf die Datenbank außerhalb von Vorlesungen oder Praktika sehr gering.
Während Praktika und Vorlesungen wird die Datenbank rege genutzt, wodurch die Performance spürbar leidet.
Dieser starke Unterschied der benötigten Leistung könnte perfekt durch Containerisierung der Datenbank ausgeglichen werden, da je nach Nachfrage innerhalb von Sekunden Container hoch- oder runtergefahren werden können.
Diese Vorgehensweise spart nicht nur Strom, sondern garantiert auch Professoren und Studierenden eine gute Performance der Datenbank. Oracle bietet im Docker Store ein vorgefertigtes Docker Image an, was die Nutzung von Oracle auf Docker sehr vereinfacht.
Eine detaillierte Anleitung für die Containerisierung von Oracle Datenbanken findet man auf der Oracle Homepage.
\footnote{\url{https://apex.oracle.com/pls/apex/germancommunities/dbacommunity/tipp/6241/index.html}}
Um die Leistungsfähigkeit der Datenbank weiter zu optimieren, könnte ein in \Abschnitt{Cluster} vorgestellter Cluster-Manager verwendet werden.
