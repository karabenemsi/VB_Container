% !TEX root = ../Ausarbeitung.tex
\section{Container und Softwareentwicklung} 
\label{sec:Softwareentwicklung}
Der Einsatz von Containern erleichtert die Entwicklung von Software in vielerlei Hinsicht.
So müssen Entwickler ihre Applikationen für verschiedene Plattformen nicht grundlegend verschieden entwerfen.
Ob für Windows, Linux, MacOS, Cloud-Plattformen oder andere, der Fokus der Entwicklung kann deutlich stärker auf die Applikation gerichtet werden, wenn die Eigenheiten der Ziel-Plattform in den Hintergrund rücken. 
Das macht den gesamten Entwicklungsprozess einfacher und somit effizienter.
Mithilfe der Abstraktion durch Container vermeidet man Inkompatibilitätsprobleme auf den Host-Geräten und auch die Entwicklung sowie Softwaretests gestalten sich dadurch leichter, schneller und effizienter, denn alle für die Applikation wichtigen Daten, Tools und Systembibliotheken sind im Container vorhanden.
Entwickler können auf verschiedenen Systemen arbeiten und unter den selben Bedingungen testen, was die Zuverlässigkeit erhöht. 
Auch ermöglichen Container die Verwendung von Microservices.
Dadurch kann sonst als monolithische Applikation entworfene Software von Entwicklern unabhängig in mehreren Teilen erstellt werden.
Außerdem ist das Software Deployment sehr simpel, da es nur gilt, ein Container Image zu erzeugen und zu verteilen.
Die Ausführung läuft auf jedem System dann jedes Mal gleich ab.

Dementsprechend benötigt man auch für Weiterentwicklung und Wartung der Applikationen weniger Zeit und Personal als wenn man für jedes System eigene Entwickler mit Fachkenntnissen bräuchte.
Bei Veränderungen an der Hardware, kurzfristigem Wechsel, Neuanschaffungen aber auch bei Upgrades des Betriebssystems hat eine Firma keine größeren Schwierigkeiten durch Inkompatibilitäten zu befürchten. Somit ist sie auch freier in der Wahl ihrer Geräte.
Auch das sogenannte Monitoring, die laufende Überwachung der Systeme, über Schnittstellen (APIs) ist mit Containern kein Problem.
Logs können von jeder Applikation erstellt, dann einfach gesammelt und in ein Management-System übertragen werden.
Die Erkennung und Eingrenzung von Fehlerquellen beschleunigt sich dadurch, dass die Applikation im Container gekapselt ist und keine weiteren Programme oder Betriebssystemteile die Fehlersuche erschweren.
Auch können die Container-Applikationen einfach neugestartet werden, sobald ein Problem erkannt wird.
Diese Vereinfachung durch Abstraktion hilft dann nicht nur dem Entwickler, sondern trägt zur Zufriedenheit der Nutzer bei. 

Seit dem Bekanntwerden von Docker im Jahr 2013 haben viele bedeutende Firmen wie Google, IBM und Netflix die Virtualisierung mit Containern eingeführt und für sich genutzt. 
Besonders wenn es darum geht, neue Applikationen zu entwerfen, deren Zielplattformen noch nicht endgültig festgelegt sind, oder bei einem Umzug in die Cloud sind Container ideal.
Insbesondere bei Cloud-Diensten sind sie aufgrund ihres geringeren Ressourcen-Umfangs beliebt.
Tools, die sich speziell um das Ressourcen-Management kümmern, sind in vielen Containern mit inbegriffen, sodass beispielsweise der zur Verfügung stehende Speicher sinnvoll begrenzt werden kann, um Out-of-memory-Abstürzen vorzubeugen. 
Das schont die Server, auf denen die Applikationen laufen, und reduziert den Hardware-Bedarf und die Kosten, wenn weniger virtuelle Maschinen mit eigenem vollwertigen Betriebssystem aufgesetzt werden müssen.