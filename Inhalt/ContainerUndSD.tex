% !TEX root = ../Ausarbeitung.tex
\section{Container in der Softwareentwicklung} 
\label{sec:Softwareentwicklung}

Der Einsatz von Containern erleichtert die Entwicklung von Software in vielerlei Hinsicht. So müssen Entwickler ihre Applikationen für verschiedene Plattformen nicht grundlegend verschieden entwerfen und die Programmiersprache kann meist frei gewählt werden. 
Ob für Windows, Linux, MacOS, Cloud-Plattformen oder andere, der Fokus der Entwicklung kann deutlich stärker auf die Funktionalitäten der Applikation gerichtet werden, wenn die Eigenheiten der Ziel-Plattform in den Hintergrund rücken. Das macht den gesamten Entwicklungsprozess einfacher und somit effizienter.
Mithilfe der Abstraktion durch Container vermeidet man Inkompatibilitätsprobleme auf den Host-Geräten und auch die Entwicklung sowie Softwaretests gestalten sich dadurch leichter, schneller und effizienter, denn alle für die Applikation wichtigen Daten, Tools und Systembibliotheken sind im Container vorhanden.
\citep{11517836120160501}

Entwickler können davon ausgehen, dass ihre Applikation auf verschiedenen Systemen funktionieren wird und können sie immer unter konsistenten Bedingungen testen, egal wie später die Umgebung aussehen mag. Dies erhöht die Zuverlässigkeit enorm. Man ist nicht länger abhängig von der Verfügbarkeit von identischen Entwicklungs- und Testsystemen und der Entwickler kann auf seinem eigenen Rechner auch schnelleres Feedback erhalten, wenn er den Container lokal ausführt und debuggt.

Auch ermöglichen Container die Verwendung von Microservices. Sonst als monolithische Applikation entworfene Software kann von Entwicklern unabhängig in mehreren Teilen erstellt werden, was die Agilität deutlich fördert.
Außerdem ist das Software Deployment sehr simpel, da es nur gilt, ein Container Image zu erzeugen und zu verteilen. Dies kann auch über Container-Orchestration-Tools wie Kubernetes nach dem Prinzip von Continous Delivery automatisiert werden. Die Ausführung läuft auf jedem System dann jedes Mal gleich ab.
\citep{ItAgil}

Dementsprechend benötigt man auch für Weiterentwicklung und Wartung der Applikationen weniger Zeit und Personal als wenn man für jedes System eigene Entwickler mit Fachkenntnissen bräuchte.
Bei Veränderungen an der Hardware, kurzfristigem Wechsel, Neuanschaffungen aber auch bei Upgrades des Betriebssystems hat eine Firma keine größeren Schwierigkeiten durch Inkompatibilitäten zu befürchten. Somit ist sie auch freier in der Wahl ihrer Geräte.

Auch das sogenannte Monitoring, die laufende Überwachung der Systeme, über Schnittstellen (APIs) ist mit Containern kein Problem. Logs können von jeder Applikation erstellt, dann einfach gesammelt und in ein Management-System übertragen werden. Die Erkennung und Eingrenzung von Fehlerquellen beschleunigt sich dadurch, dass die Applikation im Container gekapselt ist und keine weiteren Programme oder Betriebssystemteile die Fehlersuche erschweren. Auch können die Container-Applikationen einfach mit ihrem vordefinierten Idealzustand neugestartet werden, sobald ein Problem erkannt wird.
Diese Vereinfachung durch Abstraktion hilft dann nicht nur dem Entwickler, sondern trägt zur Zufriedenheit der Nutzer bei. 

Besonders wenn es darum geht, neue Applikationen zu entwerfen, deren Zielplattformen noch nicht endgültig festgelegt sind, oder bei einem Umzug in die Cloud. Gerade bei Cloud-Diensten sind Container unter anderem wegen ihres geringeren Ressourcen-Umfangs beliebt.
\citep{12771285120180201}

"Container eignen sich optimal für dienstbasierte Architekturen. Im Gegensatz zu monolithischen Architekturen, bei denen alle Teile einer Anwendung miteinander verknüpft sind […], werden diese Komponenten bei einer dienstbasierten Architektur getrennt. Durch eine Trennung und Arbeitsteilung werden Ihre Dienste auch dann weiter ausgeführt, wenn andere fehlschlagen. Damit bleibt Ihre gesamte Anwendung zuverlässiger."\citep{GoogleContainers}

Tools, die sich speziell um das Ressourcen-Management kümmern, sind in vielen Containern mit inbegriffen, sodass beispielsweise der zur Verfügung stehende Speicher sinnvoll begrenzt werden kann, um Out-of-Memory-Abstürzen vorzubeugen. 
Das schont die Server, auf denen die Applikationen laufen, und reduziert den Hardware-Bedarf und die Kosten, wenn weniger virtuelle Maschinen mit eigenem vollwertigen Betriebssystem aufgesetzt werden müssen.
\citep{11517836120160501}